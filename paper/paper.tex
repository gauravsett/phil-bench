\documentclass{article}


% if you need to pass options to natbib, use, e.g.:
%     \PassOptionsToPackage{numbers, compress}{natbib}
% before loading neurips_2023


% ready for submission
% \usepackage{neurips_2023}


% to compile a preprint version, e.g., for submission to arXiv, add add the
% [preprint] option:
\usepackage[preprint,nonatbib]{neurips_2023}


% to compile a camera-ready version, add the [final] option, e.g.:
% \usepackage[final]{neurips_2023}


% to avoid loading the natbib package, add option nonatbib:
% \usepackage[nonatbib]{neurips_2023}


\usepackage[utf8]{inputenc} % allow utf-8 input
\usepackage[T1]{fontenc}    % use 8-bit T1 fonts
\usepackage{hyperref}       % hyperlinks
\usepackage{url}            % simple URL typesetting
\usepackage{booktabs}       % professional-quality tables
\usepackage{amsfonts}       % blackboard math symbols
\usepackage{nicefrac}       % compact symbols for 1/2, etc.
\usepackage{microtype}      % microtypography
\usepackage{xcolor}         % colors
\usepackage[
    backend=biber,
    style=authoryear-comp,
    ]{biblatex}
\addbibresource{citations.bib}


\title{PhilBench: Measuring Value Learning from Text}


% The \author macro works with any number of authors. There are two commands
% used to separate the names and addresses of multiple authors: \And and \AND.
%
% Using \And between authors leaves it to LaTeX to determine where to break the
% lines. Using \AND forces a line break at that point. So, if LaTeX puts 3 of 4
% authors names on the first line, and the last on the second line, try using
% \AND instead of \And before the third author name.


\author{%
    Gaurav Sett \\ % \thanks{Use footnote for providing further information about author (webpage, alternative address)---\emph{not} for acknowledging funding agencies.} \\
    % College of Computing\\
    Georgia Institute of Technology\\
    % Atlanta, GA 30332 \\
    \texttt{gauravsett@gatech.edu} \\
  % examples of more authors
  % \And
  % Coauthor \\
  % Affiliation \\
  % Address \\
  % \texttt{email} \\
  % \AND
  % Coauthor \\
  % Affiliation \\
  % Address \\
  % \texttt{email} \\
  % \And
  % Coauthor \\
  % Affiliation \\
  % Address \\
  % \texttt{email} \\
  % \And
  % Coauthor \\
  % Affiliation \\
  % Address \\
  % \texttt{email} \\
}


\begin{document}


\maketitle


\begin{abstract}
    Current alignment efforts are largely focused on getting AI to follow common norms. 
    Strong alignment requires AI to confront complex and controversial topics. 
    Typical alignment approaches are incapable of representing a distribution of beliefs. 
    To achieve a democratic approach to alignment, we must develop new methods. 
    Collecting democratic inputs from humans at scale is expensive and difficult. 
    However, humans have already provided significant information about their values through writing. 
    We can formulate language in many ways, so the utterances we choose reveals what we think is true, informative, relevant, and coherent. 
    We propose PhilBench, a benchmark for value learning from text. We provide a dataset of text from philosophy papers. 
    We repurpose the PhilPapers Survey, measuring the views of a sample of authors of these texts, as a test for AI. 
    We encourage researchers to develop methods allowing AI to learn the values of these philosophers from their papers.
\end{abstract}


\section{Introduction}

Motivation...

\paragraph*{Insights from pragmatics.} \parencite{grice1975logic}

\paragraph*{Benchmark overview.}


\section{Data}

\subsection{Philosophy Papers}

\subsection{PhilPapers Survey}


\section{Experiments}


\section{Discussion}


\section*{Social Impacts Statement}
Potential broader impact of their work, including its ethical aspects and future societal consequences.

\section*{References}
\printbibliography[heading=none]

%%%%%%%%%%%%%%%%%%%%%%%%%%%%%%%%%%%%%%%%%%%%%%%%%%%%%%%%%%%%


\end{document}